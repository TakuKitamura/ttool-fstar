* All instructions that can be executed immediatyy are executed
actions on variables, methods, random


* All request are stored in an list of request, and added to the system
As soon as a request can be performed, it is performed

Request = timer setting, sync send, sync receive, async send, async receive

Requests of the same task are linked together


* When no request can be performed, the task terminates

* When a request is added:
The system sees whether that request can be performed immedialty -> if so, request are performed, and related tasks unblocked
Otherwise, each request is put int its wait queue


* request

#define SYNC_REQUEST 0
#define SYNC_REQUEST_WITH_DELAY 1
#define ASYNC_REQUEST 2
#define ASYNC_REQUEST_WITH_DELAY 3
#define DELAY 4

struct setOfRequest {
  request *head;
}

struct request {
  int type;
  setOfRequest* listOfRequests;
  int hasDelay;
  long delay;
  int delayElapsed;
  int [] params;
  int nbOfParams;
}


* Notion of synchronous channel
struct syncchannel {
  char *inname;
  char *outname;
  request* inWaitQueue;
  request* outWaitQueue; 
}

* Asynchronous channel
struct asyncchannel {
  char *inname
  char *outname;
  int isInfinite;
  int isBlocking;
  int maxNbOfMssages;
  request* outWaitQueue;
  request* inWaitQueue;
  messages *pendingMessages;
}

* Asynchronous message
struct Message {
  int[] params;
}



synchannel [10] syncchannels;
asynchannel [5] asyncchannels;

