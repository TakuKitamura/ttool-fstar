\documentclass[12pt]{article}
\usepackage{amsmath}
\usepackage{latexsym}
\usepackage{amsfonts}
\usepackage{amssymb}
\usepackage{graphicx}
\usepackage{txfonts}
\usepackage{wasysym}
\usepackage{adjustbox}
\usepackage{ragged2e}
\usepackage{tabularx}
\usepackage{hhline}
\usepackage{float}
\usepackage{multirow}
\usepackage{makecell}
\usepackage{fancyhdr}
\usepackage[utf8]{inputenc}
\usepackage[T1]{fontenc}
\usepackage[a4paper,bindingoffset=0.2in,headsep=0.5cm,left=1in,right=1in,bottom=3cm,top=2cm,headheight=2cm]{geometry}
\usepackage{hyperref}
\usepackage{listings}
\usepackage[most]{tcolorbox}

\lstset{language=C,basicstyle=\scriptsize\ttfamily,keywordstyle=\bfseries, commentstyle=\textit,stringstyle=\ttfamily, showspaces=false,showstringspaces=false, frame=single,
  breaklines=true,
  postbreak=\mbox{\textcolor{red}{$\hookrightarrow$}\space},
}

\everymath{\displaystyle}
\pagestyle{fancy}
\fancyhf{}
\rfoot{Page \thepage}
\begin{document}
\sloppy 

\begin{center}

\includegraphics[width=0.2\textwidth]{figures/logotpt}
\vspace{10 pt}\\
\Huge TTool \\
\vspace{10 pt}
\Large \url{ttool.telecom-paristech.fr}
\vspace{20 pt}\\
\underline{\Large SysML-Sec Tutorial}
\vspace{30 pt}
\end{center}

\begin{table}[H]
\large
\centering
\begin{adjustbox}{width=\textwidth}
\begin{tabular}{ |p{1.6cm}|p{6.0cm}|p{4.2cm}|p{4.2cm}| }
\hhline{----}
 & \textbf{Document Manager} & \textbf{Contributors}  & \textbf{Checked by}  \\ 
\hhline{----}
\textbf{Name}   & Ludovic APVRILLE & \multirow{2}{*}{Ludovic APVRILLE} &
\multirow{2}{*}{Ludovic APVRILLE} \\
\hhline{--~~}
\textbf{Contact} & ludovic.apvrille@telecom-paristech.fr &  &  \\ 
\hhline{--~~}
\textbf{Date} & \today &  &  \\ 
\hline
\end{tabular}
\end{adjustbox}
\end{table}

\newpage
\tableofcontents

% \newpage
% \listoffigures

\newpage
\section{Preface}

\subsection{Table of Versions}

\begin{table}[H]
\large
\centering
\begin{adjustbox}{width=\textwidth}
\begin{tabular}{ |p{1.5cm}|p{2.5cm}|p{9.0cm}|p{3.0cm}| }
\hhline{----}
\textbf{Version} & \textbf{Date} & \textbf{Description  $  \&  $  Rationale of
Modifications} & \textbf{Sections Modified} \\
\hhline{----}
1.0 & April 3rd, 2018 & First draft &  \\ 
\hline
\end{tabular}
\end{adjustbox}
\end{table}

\subsection{Table of References and Applicable Documents}

\begin{table}[H]
\large
\centering
\begin{adjustbox}{width=\textwidth}
\begin{tabular}{ |p{2.66in}|p{2.66in}|p{0.95in}|p{0.43in}| }
\hhline{----}
\textbf{Reference} & \textbf{Title  $  \&  $  Edition} & \textbf{Author or
Editor} & \textbf{Year}
\\
\hhline{----}
 &  &  &  \\ 
\hline
\end{tabular}
\end{adjustbox}
\end{table}

\subsection{Acronyms and glossary}

\begin{table}[H]
\large
\centering
\begin{adjustbox}{width=\textwidth}
\begin{tabular}{ |p{1.24in}|p{5.45in}| }
\hhline{--}
\textbf{Term} & \textbf{Description} \\ 
\hhline{--}
 &  \\ 
\hline
\end{tabular}
\end{adjustbox}
\end{table}

\subsection{Summary}

This document describes how to use SysML-Sec using simple examples\footnote{This document has been started in the scope of the AQUAS european project}. In particular, it covers requirements, attack trees, HW/SW partitioning and  software design.

\newpage

\section{Configuration}\label{sec:conf}
\subsection{TTool configuration}
At first, if not already configured\footnote{You version of TTool should be already configured}, you must open the configuration file of TTool. The default file is located in:
\begin{verbatim}
TTool/bin/config.xml
\end{verbatim}
Open your configuration file, and set the following lines accordingly with your TTool and ProVerif installation:
\begin{itemize}
\item Directory in which formal specifications for security proofs are generated:
\begin{verbatim}
<ProVerifCodeDirectory data="../proverif/" />
\end{verbatim}
\item Path to the proverif executable file:
\begin{verbatim}
<ProVerifVerifierPath data="/opt/proverif/proverif" />
\end{verbatim}
\item Host on which the proof will be started (for example, you could execute this proof on a dedicated machine if the "ProVerifCodeDirectory" is reachable from that dedicated machine):
\begin{verbatim}
<ProVerifVerifierHost data="localhost" />
\end{verbatim}
\end{itemize}
We also assume that the DIPLODOCUS simulator correctly works.


\subsection{External tools}
The  configuration for the DIPLODOCUS simulator assumes that a \textbf{C compiler}, referenced by the provided Makefile (default = "gcc"\footnote{\url{https://gcc.gnu.org/}}) is installed on your machine, as well as the \textbf{POSIX-1 librairies}. Also, a Makefile utility must be installed (e.g., "GNU make"\footnote{\url{https://www.gnu.org/software/make/}}).

\newpage
\section{A toy example}\label{sec:example}
This very first example explains how to use the main capabilities of SysML-Sec

\subsection{Getting the example}
Be sure to get the latest version of TTool including the remote loading of models (March 2018 and after). Do: File, Open from TTool repository, and select "SysMLSecTutorial.xml".

\subsection{Understanding the model}
The first tab of the model presents an overview of the SysML-Sec methodology (see Figure \ref{fig:method}). Each stage of the method is represented with a rectangle that contains a link to the corresponding diagrams. 


\begin{figure*}[htbp]
\centering
\includegraphics[scale=0.8]{build/method-svg}

\caption{The first diagram represents the method of SysML-Sec. Each stage of the method is represented by a rectangle that contains a link to all diagrams of the corresponding stage} \label{fig:method}
\end{figure*}


\end{document}
